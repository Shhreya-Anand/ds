\documentclass[]{article}

\usepackage{tikz}
\usepackage{float}
\usetikzlibrary{arrows,shapes.geometric}

\usepackage{amsmath}
\usepackage{amssymb}
\usepackage{amsthm}

\newtheorem{theorem}{Theorem}[section]
\newtheorem{algorithm}[theorem]{Algorithm} %{\bfseries}{\itshape}
\newtheorem{lemma}[theorem]{Lemma}
\newtheorem{note}[theorem]{Note}
\newtheorem{proposition}[theorem]{Proposition}
\newtheorem{corollary}[theorem]{Corollary}

\newenvironment{definition}[1][Definition]{\begin{trivlist}
\item[\hskip \labelsep {\bfseries #1}]}{\end{trivlist}}
\newenvironment{example}[1][Example]{\begin{trivlist}
\item[\hskip \labelsep {\bfseries #1}]}{\end{trivlist}}
\newenvironment{remark}[1][Remark]{\begin{trivlist}
\item[\hskip \labelsep {\bfseries #1}]}{\end{trivlist}}

\newcommand{\Oo}{\mathcal O} %big-O notation
\newcommand{\oo}{\mathcal o} %small-O notation ;)

\title{Chain hashing for variable length strings}
\author{Thomas Laumann Jespersen \and Eva Rotenberg}

\begin{document}
\maketitle

\section{Theory}

Given a string $x=x_0 \ldots x_d$ with $x_i < u$, and a codomain $[m]$, let $p$ be a small Mersenne prime larger than $u$ and $m$. We use the following hash function, given random seeds $a,b < p$:
\[h(x)= \left( b \left( \sum_{i=0}^{d}x_i a^i \right) \mod p \right) \mod m \]

Given that $d$ is no larger than $p/m$, the collision probability is at most $2/m$. Setting $p=2^{89}-1$ and $u=2^{64}$ and $m=2^{32}$, this means we are safe for strings of length up $2^{57}$, which is enough for our purposes.

Our algorithm thus states roughly:

\begin{verbatim}
a <- random number
b <- random number
ai = 1
res = x[0]
for(i = 1 to d){
  ai = ai * a (mod p)
  res = res + ai * x[i] (mod p)
}
res = res * b (mod p)
res = res mod p
\end{verbatim}

As earlier noted, we can use the congruence $x \equiv x\mod 2^q + \lfloor x / 2^q \rfloor$, to make quicker calculations modulo a Mersenne prime, $2^q -1$.

In order to multiply $>32$ bit numbers, in fact several multiplications are needed. When multiplying two circa $89$ bit numbers, we split each number into two parts: $a = a_2\cdot 2^{64} + a_1 \cdot 2^{32} + a_0$, and similarly, $y = y_2\cdot y^{64} + y_1 \cdot 2^{32} + y_0$, where $a_i$ and $y_j$ are all $32$-bit numbers. Then, we calculate all the multiplications $a_i y_j$. To find the mod p value of $ay$, we analyse each summand, which is on the form $a_i y_j \cdot 2^{k}$, where $a_i y_j$ is a $64$ bit number. Using the congruence above, when $k>q$, we do not need to add the first summand $a_i y_j \cdot 2^k \mod 2^q = 0$, and need only add $a_i y_j \cdot 2^{k-q}$. Symmetrically, $a_0 y_0$ only contributes with the first summand, as $a_0 y_0 < 2^q$.

\section{Experiment}
We compare our function with the c++ standard library. As objects we have downloaded books in different languages from The Gutenberg Project. 

\end{document}
