\newtheorem{note}[theorem]{Note}
\newtheorem{proposition}[theorem]{Proposition}
\newtheorem{corollary}[theorem]{Corollary}

\newenvironment{definition}[1][Definition]{\begin{trivlist}
\item[\hskip \labelsep {\bfseries #1}]}{\end{trivlist}}
\newenvironment{example}[1][Example]{\begin{trivlist}
\item[\hskip \labelsep {\bfseries #1}]}{\end{trivlist}}
\newenvironment{remark}[1][Remark]{\begin{trivlist}
\item[\hskip \labelsep {\bfseries #1}]}{\end{trivlist}}

\newcommand{\Oo}{\mathcal O} %big-O notation
\newcommand{\oo}{\mathcal o} %small-O notation ;)

\title{Hashing for variable length strings}
\author{Thomas Laumann Jespersen \and Eva Rotenberg}

\begin{document}
\maketitle

\section{Theory}

\subsection*{Direct polynomial string hashing}

Given a string $x=x_0 \ldots x_d$ with $x_i < u$, and a codomain $[m]$, let $p$ be a small Mersenne prime larger than $u$ and $m$. We use the following hash function, given random seeds $a,b < p$:
\[h(x)= \left( b \left( \sum_{i=0}^{d}x_i a^i \right) \mod p \right) \mod m \]

Given that $d$ is no larger than $p/m$, the collision probability is at most $2/m$. Setting $p=2^{89}-1$ and $u=2^{64}$ and $m=2^{32}$, this means we are safe for strings of length up $2^{57}$, which is enough for our purposes.

Our algorithm thus states roughly:

\begin{verbatim}
a <- random number
b <- random number
ai = 1
res = x[0]
for(i = 1 to d){
  ai = ai * a (mod p)
  res = res + ai * x[i] (mod p)
}
res = res * b (mod p)
res = res mod p
\end{verbatim}

As earlier noted, we can use the congruence $x \equiv x\mod 2^q + \lfloor x / 2^q \rfloor$, to make quicker calculations modulo a Mersenne prime, $2^q -1$.

In order to multiply $>32$ bit numbers, in fact several multiplications are needed. When multiplying two circa $89$ bit numbers, we split each number into two parts: $a = a_2\cdot 2^{64} + a_1 \cdot 2^{32} + a_0$, and similarly, $y = y_2\cdot y^{64} + y_1 \cdot 2^{32} + y_0$, where $a_i$ and $y_j$ are all $32$-bit numbers. Then, we calculate all the multiplications $a_i y_j$. To find the mod p value of $ay$, we analyse each summand, which is on the form $a_i y_j \cdot 2^{k}$, where $a_i y_j$ is a $64$ bit number. Using the congruence above, when $k>q$, we do not need to add the first summand $a_i y_j \cdot 2^k \mod 2^q = 0$, and need only add $a_i y_j \cdot 2^{k-q}$. Symmetrically, $a_0 y_0$ only contributes with the first summand, as $a_0 y_0 < 2^q$. For $k<q$ we have the equivalence $a_i y_j 2^k \equiv (a_i y-_j \mod 2^{q-k})\cdot 2^k + a_i y_j / 2^{q-k}$.

\[\begin{matrix}
& & (a_2 2^{64} + a_1 2^{32} + a_0)\cdot (y_2 2^{} + y_1 2^{32} + y_0) \\
& \equiv & a_2 y_2 \cdot 2^{39} + [a_2 y_1,a_1 y_2] \cdot 2^7 + \ldots + a_0 y_0  \\
& \equiv & a_2 y_2 \mod 2^{50} \cdot 2^{31} + a_2 y_2 / 2^{50} + [a_1 y_2, a_2 y_1] \cdot 2^7 \\
& + & [a_2 y_0, a_1 y_1, a_0 a_2] \mod 2^{25} \cdot 2^{64} + [a_2 y_0, a_1 y_1, a_0 a_2] /2^{25} \\
& + & [a_1 y_0, a_0 y_1] \mod 2^{57} \cdot 2^{32} + [a_1 y_0, a_0 y_1] /2^{57} + a_0 y_0
\end{matrix}
\]

Using the calculation above, we can obtain a sum of $89$-bit numbers. If we also perform modulo on the high-part of the sum, we obtain a $\le 90$-bit number, which leaves us satisfied.

\subsection*{Wrapped polynomial string hashing}

The algorithm above runs too slowly for large strings. To speed it up, we consider the following "wrapper". 

Given a fast universal hash function
\[r : [2^{64}]^d \to [2^{64}] \]

We can divide the input into chunks $X_0 \ldots X_j$, such that each $X_i$ contains $32$ integers of $64$ bits for $i<j$, and $X_j$ contains at most $32$ integers. Then apply $r$ to each $X_i$, $i=0\ldots j$, obtaining the sequence $\left(r(X_0), \ldots , r(X_j)\right)$. This sequence is a much shorter string, to which we can apply polynomial string hashing.

As a fast universal hash function, we concatinate two versions, $r_1$ and $r_2$ of pair-multiply-shift:
\[r_1(x_0,\ldots ,x_{d-1}) = \left(\left(\sum_{i\in [d/2]}(a_{2i} + x_{2i})(a_{2i+1} + x_{2i+1}) \right)+b \right) [\bar{w}-l,\bar{w}]\]
\[r_2(x_0,\ldots ,x_{d-1}) = \left(\left(\sum_{i\in [d/2]}(\alpha_{2i} + x_{2i})(\alpha_{2i+1} + x_{2i+1}) \right)+\beta \right) [\bar{w}-l,\bar{w}]\]
Where $a_i,b_i,\alpha_i,\beta_i$ are random integers, and where $n[\bar{w}-l,\bar{w}]$ is the $l$ bit number consisting of bit $\bar{w}-l$ to bit $\bar{w}$ of the number $n$.

\section{Implementation}

\paragraph{Number struct}
\paragraph{Direct PSH} 
\paragraph{Pair-multiply-shift}
\paragraph{Wrapped PSH}

\section{Unit testing}

\paragraph{Multiply mod $p = 2^{81} -1$.} To test correctness of out mod-$p$-multiplier, we multiply two random numbers $a,b$ using a black-box multiplication function we assume to work obtaining the product $a*b$. Then we calculate our mod-$p$ product, $a\cdot b$. If the difference is divisible by $p$, we are happy. We calculate: 
$$\left((a*b - a\cdot b)\mod p\right) ~ \stackrel{?}{=} ~ 0.$$

%We test several cases: a $32$-bit number with a $32$-bit number, with and without over flow; a $32$-bit number with a $64$-bit number, 

\section{Experiment}
We compare our function with the c++ standard library. As objects we have downloaded books in different languages from The Gutenberg Project. 

\end{document}
